%!TEX root = dissertation.tex

\chapter*{About this project}
\paragraph{Abstract}

Dúchas Na Gaillimhe Galway Civic Trust along with tourism Ireland carry out walking tours of Galway city to inform tourist of Galways history. One of the walks they offer tourist during the summer months is a medieval walking tour of Galway city. This walk visits seven locations starting at Galways Latin Quarter.However This limits tourist to attend lead tours at certain hours as well as only being available frequently during the summer tourism season.

This project aims to deliver a solution which helps the tourist of Galway city carry of the medieval walking tour of Galway city without the 
need of a guide and available at any time of the year. If the tourist wishes to seek further information they can drop into Dúchas Na Gallimhe Galway Civic Trust office which is located at one of the stops,The Hall Of the Red Earl.

The proposed solution will be comprised of a mobile application which will allow users to be guided around Galway city. It will show the user their location vs. the destination they must reach.On arrival to the destination, give the used information and images on the location. A user can find out extra information if they wish. There will also be a web application used by Galway Civic Trust to update the information on the application public to all users.

\paragraph{Authors}
Sarah Carroll, Abigail Culkin
B.Sc.(Hons) in Software Development

\chapter{Introduction}

In the beginning of Year 4 in Software Development at GMIT, the authors were given the opportunity to work on the Medieval walking tour project for an external small/medium Enterprise organised through GMIT. The idea of the project is based on the problem that medieval walking tours of galway city are not always available for tourists. 

The Project intends to make walking tours of Galway city more accessible and easy to follow for tourists. This will up tourism in Ireland along with bring more visitors to the hidden wonders of Galway city.
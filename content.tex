%!TEX root = dissertation.tex

\chapter*{About this project}
\paragraph{Abstract}

Nutritional information is provided on products using different size metrics such as Per 100g serving, Per Unit (whole item) serving and in some cases a serving size suggestion. Due to a need to compute and calculate the nutritional benefit of a product, consumers can in some cases be misled into getting a less healthy product based on how the information and sizes are presented on each product. Furthermore, consumers trying to make purchase decisions may not have the time or ability to calculate and compare two products sugar content during a shopping trip.

This project aims to deliver a solution which helps the general public to better comprehend the nutritional benefit of their product choices by breaking down nutritional metrics like the sugar content of a product and presenting the data in a format which is easier to visualise than adding up grams. In addition to this, with access to data of other food and drink products, suggestions can be provided to the user which are similar types of products but have less sugar. 
The proposed solution will be comprised of a mobile application which will allow users to scan food and drink product's barcodes and receive suggestions of similar products with a lower sugar count. Values for both the scanned product and suggestions sugar content will be displayed.
\paragraph{Authors}
Sarah Carroll, Abigail Culkin
B.Sc.(Hons) in Software Development



\chapter{Introduction}
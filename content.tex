%!TEX root = dissertation.tex

\chapter*{About this project}
\paragraph{Abstract}

Dúchas Na Gaillimhe Galway Civic Trust along with tourism Ireland carry out walking tours of Galway city to inform tourist of Galways history. One of the walks they offer tourist during the summer months is a medieval walking tour of Galway city. This walk visits seven locations starting at Galways Latin Quarter.However This limits tourist to attend lead tours at certain hours as well as only being available frequently during the summer tourism season.

This project aims to deliver a solution which helps the tourist of Galway city carry of the medieval walking tour of Galway city without the 
need of a guide and available at any time of the year. If the tourist wishes to seek further information they can drop into Dúchas Na Gallimhe Galway Civic Trust office which is located at one of the stops,The Hall Of the Red Earl.

The proposed solution will be comprised of a mobile application which will allow users to be guided around Galway city. It will show the user their location vs. the destination they must reach.On arrival to the destination, give the used information and images on the location. A user can find out extra information if they wish. There will also be a web application used by Galway Civic Trust to update the information on the application public to all users.

\paragraph{Authors}
Sarah Carroll, Abigail Culkin
B.Sc.(Hons) in Software Development

\chapter{Introduction}

In the beginning of Year 4 in Software Development at GMIT, the authors were given the opportunity to work on the Medieval walking tour project for an external small/medium Enterprise organised through GMIT. The idea of the project is based on the problem that medieval walking tours of galway city are not always available for tourists. 

The Project intends to make walking tours of Galway city more accessible and easy to follow for tourists. This will up tourism in Ireland along with bring more visitors to the hidden wonders of Galway city.

\chapter{Context}

\section{Context of the Project}

The context of this project revolves around the use case of being a tourist in galway city, wishing to find out more information about the cities history. Opening the app on your phone,viewing your location is relation to the spots on the guided tour. This application prevents the user getting lost and gives them accurate information about each location.  The information on the application was provided by Galway Civic trust and is the same information tourist get on the guided walking tours. Reading the data and retrieving images from the database needs to be very fast, as any delayed hang in performing could cause a bad user experience.

The project will be developed as a Cross-Platform application which provides users with images, information and a map of Galway city and the 5 medieval stops along the tour. Users of the application can get their location in relation to the exact location of the points of the tour at any given time. When the user decides they wish to learn more information about a specific location, the user will get images and text about the location on the application. Along with the application a web application will be developed for use by galway civic trust to update delete or add information to the application.

\section{Objectives} 


The project will require a number of objectives to be accomplished in order to provide a solution that works and is suitable for the use of Galway Civic Trust. 

\begin{itemize}
\item A Mongo database will be used to store the text and images which appear in the application. This database will need to be setup and hosted in such a way that it can be accessed from clients on the web. The database will be hosted on Amazon Web Services Cloud.

\item A client application will be the main product / asset for the project. This application will be able to locate the user via the gps on the mobile. The app will then show the user the locations of each point on the tour. This will get the user an idea of how long they are from the point and will allow then to see when they have reached their location.

\item Another requirement is to develop a web application for the use of the company to allow them to modify the information shown to the user.

\item Use google maps Api within the mobile application to show the user their location along with the specific points of the tour.

\item Pull information from the mongo db using node.js


\end{itemize}
\section{Overview}

Each Chapter of this paper will contain different details regarding the project. 

The Methodologies will describe the way in which decisions were made about the project. The way in which software development and research methodology was addressed. The aids used to help working in a team will also be discussed.

The technology review will go into more detail on the different technologies used throughout the project and the reasoning for choosing each. An example of these technologies are node.js.

The System Design will provide a detailed explanation of the overall system architecture. The how of the project.

The System Evaluation compares the project against the initial objectives set out in the introduction. This will also include how new features could be included for the customer.

The conclusing briefly summerises the context and objectives of the project.
 



\chapter*{About this project}
\paragraph{Abstract}

Dúchas Na Gaillimhe ( Galway Civic Trust ) along with Tourism Ireland carry out walking tours of Galway city to inform visitors of Galway's long and rich history. One of the walks they offer tourists during the summer months is a medieval walking tour of Galway city. This walk visits seven locations starting in Galways Latin Quarter. However, this limits visitors to attending guide-lead tours at specific times as well as only being available frequently during the summer tourist season.

This project aims to deliver a solution which helps tourists visiting Galway city to take the medieval walking tour of Galway at any time, without the need of a guide and will be available at any time of the year. If the tourist wishes to seek further information they can drop into Galway Civic Trust's office which is located at one of the seven stops along the route - at the Hall Of the Red Earl.

The proposed solution will comprise a mobile phone application which will allow users to be guided around Galway city. It will show the user their current location and guide them to the destination they wish to reach. On arrival at the destination, the app provides the user with information and images about the location. A user can get additional information if they so wish. There will also be a web application used by Galway Civic Trust personnel to update the information on the application periodically and make it automatically available to all new and existing app users.

\paragraph{Authors}
Sarah Carroll, Abigail Culkin
B.Sc.(Hons) in Software Development

\chapter{Introduction}

At the beginning of Year 4 in Software Development at GMIT, the authors were given the opportunity to work on the Medieval walking tour project for an external small/medium Enterprise organised through GMIT. The idea of the project is based on the problem that medieval walking tours of Galway city are not always available for tourists at convienient times and infrequently outside the main summer season. 


The purpose of this Final Year Project is to build an application using new technologies that the developers have never used before. Therefore, it allows for research to be done and learn up and coming technologies and learn new ways of developing software. The front-end technology being used is Flutter. It will be connected to a backend server running on Google cloud platform and within this a MongoDB Flask Database which uses docker. Working with Galway Civic trust allowed for an application to be made that is missing from the market place for Galway tourism. This application will be useful and easy to use for customers and the company will benefit from this.  This way the authors will be able to show they have worked with an outside source and have a fully functioning app that will satisfy what was asked of them.


It will be explained the technologies used and why we used them throughout. The research and development will all be detailed and allow others to understand why specific software was chosen. This project is done using technologies that have not been used together too often. This means any problems faced or certain aspects found to be new will be documented along with the development.
\section{Context}
